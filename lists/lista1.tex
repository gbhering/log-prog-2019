\documentclass[11pt]{article}

%Gummi|065|=)
\usepackage[T1]{fontenc}
\usepackage[utf8]{inputenc}
\usepackage{amsmath,wasysym}
\usepackage{tikz}
\usetikzlibrary{arrows,automata}
\usepackage[portuguese]{babel}

%silly things to make my life easier
\newcommand{\sq}{\Square \,}
\newcommand{\di}{\Diamond \,}
\newcommand{\imp}{\rightarrow}
\newcommand{\F}{\mathcal{F}}
\newcommand{\M}{\mathcal{M}}
\newcommand{\mwm}{\mathcal{M}, w \models \;}
\newcommand{\mwn}{\mathcal{M}, w \not \models \;}
\newcommand{\spcmw}{Prova: suponha, por contradição, $\M = (\F, V)$, com um $w \in W$ tal que}

\title{\textbf{Lista sobre Lógica Modal}}
\author{Gabriel Bhering Dominoni}
\date{\today}
\begin{document}

\maketitle

\section{Mostre que as seguintes fórmulas são válidas ou não na classe $\F$ de todos os frames}




% 1.1
\subsection{$ \F\models \sq ( \phi \land \psi ) \imp ( \sq \phi \land \sq \psi ) $ | válida}
Intuição: Se todos vizinhos tem $\phi$ e $\psi$, todos vizinhos tem $\phi$ e todos vizinhos tem $\psi$. Se sabemos que $\phi$ e $\psi$, sabemos que $\phi$ e sabemos que $\psi$. \\

\spcmw
\begin{flalign} 
\mwn & \sq ( \phi \land \psi ) \imp ( \sq \phi \land \sq \psi ) \\
\mwm & \sq ( \phi \land \psi ) \label{1.1.1} \\
\mwn & \sq \phi \land \sq \psi \label{1.1.2}
\end{flalign}

Contradição, por (\ref{1.1.1}) e (\ref{1.1.2}): (\ref{1.1.1}) é válido sse todo vizinho de $w$ tem tanto $\phi$ quanto $\psi$, mas (\ref{1.1.2}) só é válido se nem todo vizinho tem $\phi$ ou se nem todo vizinho tem $\psi$.



% 1.2
\subsection{$ \F\models ( \sq \phi \land \sq \psi ) \imp \sq ( \phi \land \psi ) $  | válida}
Intuição: Se todos vizinhos tem $\phi$ e todos vizinhos tem $\psi$, todos vizinhos tem $\phi$ e $\psi$. Se sabemos que $\phi$ e sabemos que $\psi$, sabemos que $\phi$ e $\psi$.\\

\spcmw
\begin{flalign} 
\mwn & ( \sq \phi \land \sq \psi ) \imp \sq ( \phi \land \psi ) \\
\mwm & \sq \psi \label{1.2.1} \\
\mwm & \sq \phi \label{1.2.2} \\
\mwn & \sq ( \phi \land \psi ) \label{1.2.3} 
\end{flalign}

Contradição, por (\ref{1.2.1}), (\ref{1.2.2}) e (\ref{1.2.3}): (\ref{1.2.3}) é válido sse nem todos os vizinhos de $w$ tem tanto $\phi$ quanto $\psi$, porém por \ref{1.2.2} e \ref{1.2.3} temos que todos os vizinhos de $w$ tem, respectivamente, $\phi$ e $\psi$.



% 1.3
\subsection{$ \F\models \di ( \phi \land \psi ) \imp ( \di \phi \land \di \psi ) $ | válida}
Intuição: Se algum vizinho tem $\phi$ e $\psi$, pelo menos aquele vizinho tem $\phi$ e pelo menos aquele vizinho tem $\psi$. Se acreditamos em $\phi$ e $\psi$, acreditamos em $\phi$ e acreditamos em $\psi$.\\

\spcmw
\begin{flalign} 
\mwn & \di ( \phi \land \psi ) \imp ( \di \phi \land \di \psi ) \\
\mwm & \di ( \phi \land \psi ) \label{1.3.1} \\
\mwn & \di \phi \land \di \psi \label{1.3.2} 
\end{flalign}

Contradição, por (\ref{1.3.1}) e (\ref{1.3.2}): (\ref{1.3.1}) nos dá que algum vizinho de $w$ tem $\phi$ e $\psi$, porém (\ref{1.3.1}) é válido sse nenhum vizinho tem $\phi$ ou nenhum vizinho tem $\psi$.



% 1.4 
\subsection{$ \F\models ( \di \phi \land \di \psi ) \imp \di ( \phi \land \psi ) $ | inválida}
Intuição: É possível que algum vizinho tenha $\phi$ e algum vizinho tenha $\psi$, mas não necessáriamente algum vizinho tem ambos. Se acreditamos na possibilidade de $\phi$ e na possibilidade de $\psi$, não necessáriamente acreditamos na possiblidade de ambos.\\

Prova:
\begin{figure}[!h]
\centering
\begin{tikzpicture}[->,>=stealth',shorten >=2pt,auto,node distance=2cm, semithick]
  \tikzstyle{every state}=[fill=white,text=black]

  \node[state] 		   (A)                   {$w_1$};
  \node[state]         (B) [left of=A]       {$w_2$};
  \node[state]         (C) [right of=A]      {$w_3$};
  \node	(VB)          [left of=B,xshift=12mm]      {$\phi$};
  \node (VC)        [right of=C,xshift=-12mm]      {$\psi$};

  \path (A) edge              node {} (B)
            edge              node {} (C);
\end{tikzpicture}
\end{figure}

$\M_{1.4} = ( \{w_1, w_2, w_3\}, \; w_1Rw_2, \; w_1Rw_2, \; V(\phi) = \{w_2\}, \; V(\psi) = \{w_3\}) $ 



% 1.5
\subsection{$ \F\models \sq ( \phi \lor \psi ) \imp ( \sq \phi \lor \sq \psi ) $ | inválida}

Intuição: Similar a anterior, é possível que todos vizinhos tenham $\phi$ ou $\psi$, mas não necessáriamente todos vizinhos tem $\phi$ ou todos tem $\psi$. Sabemos que ou $\phi$ ou $\psi$... mas não necessariamente sabemos algum/qual deles. \\

Prova: mesmo modelo $\M_{1.4}$ acima.



% 1.6
\subsection{$ \F\models ( \sq \phi \lor \sq \psi ) \imp \sq ( \phi \lor \psi ) $}
Intuição: Se todos vizinhos tem $\phi$, todos tem $\phi$ ou $\psi$; ou se todos vizinhos tem $\psi$, então todos tem $\phi$ ou $\psi$ \\

\spcmw
\begin{flalign} 
\mwn & ( \sq \phi \lor \sq \psi ) \imp \sq ( \phi \lor \psi ) \\
\mwm & \sq \phi \lor \sq \psi \label{1.6.1} \\
\mwn & \sq ( \phi \lor \psi ) \label{1.6.2} 
\end{flalign}

Contradição por (\ref{1.6.1}) e (\ref{1.6.2}): (\ref{1.6.2}) é válida sse nem todos os vizinhos de $w$ não tem nem $\phi$ nem $\psi$, mas por (\ref{1.6.1}) temos que ou todos os vizinhos tem $\phi$ ou todos os vizinhos tem $\psi$. 



% 1.7
\subsection{$ \F\models \di ( \phi \lor \psi ) \imp ( \di \phi \lor \di \psi ) $}
Intuição: Se algum vizinho tem $\phi$ ou tem $\psi$, então pelo menos ele tem $\phi$, ou pelo menos ele tem $\psi$ \\

\spcmw
\begin{flalign} 
\mwn & \di ( \phi \lor \psi ) \imp ( \di \phi \lor \di \psi ) \\
\mwm & \di ( \phi \lor \psi ) \label{1.7.1}\\
\mwn & \di \phi \label{1.7.2} \\
\mwn & \di \psi \label{1.7.3} 
\end{flalign}

Contradição por (\ref{1.7.1}), (\ref{1.7.2})  e (\ref{1.7.3}): Por (\ref{1.7.1}), temos que algum vizinho de $w$ tem $\phi$ ou tem $\psi$, mas por (\ref{1.7.2}) temos que nenhum vizinho tem $\phi$, e por (\ref{1.7.3}) que nenhum vizinho tem $\psi$, logo as três não podem ser válidas ao mesmo tempo.



% 1.8
\subsection{$ \F\models ( \di \phi \lor \di \psi ) \imp \di ( \phi \lor \psi ) $}
Intuição: Dado que ou algum vizinho tem $\phi$ ou algum vizinho tem $\psi$, esse(s) vizinho(s) tem $\phi$ ou tem $\psi$.




\subsection{$ \F\models \sq ( \phi \imp \psi ) \imp ( \sq \phi \imp \sq \psi ) $}




\subsection{$ \F\models ( \sq \phi \imp \sq \psi ) \imp \sq ( \phi \imp \psi ) $}




\subsection{$ \F\models \di ( \phi \imp \psi ) \imp ( \di \phi \imp \di \psi ) $}




\subsection{$ \F\models ( \di \phi \imp \di \psi ) \imp \di ( \phi \imp \psi ) $}




\subsection{$ \F\models \di \phi \imp \lnot \sq \lnot \phi $}




\subsection{$ \F\models \sq \phi \imp \lnot \di \lnot \phi $}



\end{document}
